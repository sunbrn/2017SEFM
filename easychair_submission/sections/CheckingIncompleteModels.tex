\emph{Partial Kripke Structures}~\cite{bruns1999model}  (PKSs) extend KSs by allowing a proposition in a given state to be labelled with $?$ to represent an unknown value. 
A PKS $M$ is a tuple $\langle S, R,S_0,AP,L \rangle$,
where:
\begin{enumerate*}
\item[] $S$ is a set of \emph{states};
\item[] $R\subseteq S\times S$ is a \emph{left-total} \emph{transition relation} on $S$;
\item[] $S_0$ is a set of initial states;
\item[] $AP$ is a set of atomic propositions;
\item[] $L: S\times AP\rightarrow \{\top,?,\bot\}$ is a \emph{function} that, for each state in $S$, associates a truth value in the set $\{\top,?,\bot\}$ to every atomic proposition in $AP$.
\end{enumerate*}  
The model of the grade crossing semaphore presented in Figure~\ref{fig:modelmot}  is an example of a PKS.

A \emph{completion} of a PKS $M$ is a KS $M^\prime$ that completes $M$ by assigning values to the unknown propositions.
The set $\mathcal{C}(M)$ contains all the completions of $M$.

Two kinds of LTL semantics (three valued and thorough) exist for  PKSs. 
%When the satisfaction of LTL formulae is considered w.r.t. PKSs, either the three-valued or the thorough semantics can be considered.
%----------------------------------------------------------------------------------------------------------------------------------------------------------------
% Three-valued LTL semantic
%----------------------------------------------------------------------------------------------------------------------------------------------------------------


\emph{Three-valued LTL semantics} $[(M,\pi) \models \phi]$ associates to a model $M$, a path $\pi$ of  $M$, and a formula $\phi$, a truth value in the set $\{ \bot, ?, \top \}$. This semantics specifies that a formula $\phi$ definitely holds in a PKS $M$ if it is true for all possible values of the unknown propositions in $M$. 
Likewise, it is definitely violated if it is false despite the unknown values.
%A formula $\phi$ is not satisfied if there is a path $\pi$ in the PKS which violates $\phi$ despite the unknown values of the atomic propositions associated to the states of $\pi$.
%Otherwise the formula is unknown.
According to three-valued semantics~\cite{godefroid2011ltl}, given a PKS $M = \langle S, R,$ $S_0, AP, L \rangle$, a path $\pi=s_0,s_1,\ldots$, and a formula $\phi$, we inductively define that $\pi$ satisfies $\phi$ in the model $M$ as follows:
\begin{align*}
&[(M,\pi) \models p] &   = &&& L(s_0,p)\\
&[(M,\pi) \models \lnot\phi] &   = &&& \textnormal{comp}([(M,\pi) \models\phi]) \\
&[(M,\pi) \models \phi_1 \LTLand \phi_2] &   =   &&& \min([(M,\pi) \models\phi_1],[(M,\pi) \models\phi_2])\\
&[(M,\pi) \models \LTLnext \phi] &   =  &&& [(M,\pi^1) \models\phi]\\
&[(M,\pi) \models \phi_1 \LTLuntil \phi_2] &   =  &&& \max_{j\geq 0}(\min(\{[(M,\pi^i) \models\phi_1]|i<j\} \cup \{[(M,\pi^j) \models\phi_2]\}))
\end{align*}
\noindent{where the notation $\pi^i$ indicates the sub-path $s_i, s_{i+1} \ldots $ of $\pi$.}

Negation is defined by the function comp (complement), which maps $\top$ to $\bot$, $\bot$ to $\top$, and $?$ to $?$.
The conjunction (disjunction) is defined as the minimum (maximum) of its arguments, following the order $\bot <\ ? < \top$. These functions are extended to sets considering min($\emptyset$)=$\top$ and max($\emptyset$)=$\bot$.

Given a PKS $M = \langle S, R,$ $S_0, AP, L \rangle$, satisfaction of formula $\phi$ in a state $s$ is defined as  $[(M, s) \models \phi]  =   \min(\{[(M, \pi) \models \phi] \mid \pi^0=s\})$.
A PKS $M$ \emph{definitely satisfies} a property $\phi$ ($[M \models \phi]=\top$) iff for all initial states $s_0 \in S_0$ of $M$, $[(M, s_0) \models \phi]=\top$. 
A PKS $M$ \emph{does not satisfy} the property $\phi$ ($[M \models \phi]=\bot$) iff there exists an initial state $s_0 \in S_0$ of $M$ such that  $[(M, s_0) \models \phi]=\bot$.
A PKS \emph{possibly satisfies} $\phi$ otherwise.
%In the rest of the paper we will use interchangeably the expressions ``$M$ definitely satisfies $\phi$'' and ``$\phi$ holds in $M$'', ``$M$ does not satisfy $\phi$'' and ``$\phi$ does not hold in $M$'', as well as ``$M$ possibly satisfies $\phi$'' and ``$\phi$ possibly holds in $M$''.



Three-valued semantics does not behave always in accordance with the natural intuition~\cite{bruns2000model}: there are cases in which $\phi$ possibly holds for a PKS but all its  completions 
%(obtained by replacing the $?$ values with $\top$ and $\bot$) 
actually satisfy (or do not satisfy)  $\phi$.
%For instance, this happens when $\phi$ is a tautology or it is unsatisfiable with a traditional two-valued interpretation. 
For this reason, an alternative semantics, called \emph{thorough LTL semantics}~\cite{bruns2000model} has been proposed. 
According to it, a formula is possibly satisfied only if there exist two completions $M_1, M_2 \in \mathcal{C}(M)$, such that $\phi$ is definitely satisfied in one and violated in the other.
%Thorough semantics defines satisfaction of an  LTL formula $\phi$ by a PKS $M$ ($[M\models\phi]_t$)  as follows:
Thorough semantics defines satisfaction of an  LTL formula $\phi$ by a PKS $M$ as follows:
\begin{equation}
 [M \models\phi]_t  = 
                						\begin{cases}
                  								\top & \quad  \text{if}\ M^\prime \models \phi \text{ for all } M^\prime \in \mathcal{C}(M)\\
                  								\bot & \quad  \text{if}\ M^\prime \not\models\phi \text{ for all } M^\prime \in \mathcal{C}(M)\\
                  								? &  \quad  \text{otherwise}
                							  \end{cases}      					  		   \nonumber
\end{equation}


Given a PKS and an LTL formula $\phi$, it has been proved~\cite{godefroid2011ltl} that 
\begin{enumerate*}[label={(\arabic*)}]
\item $[M\models\phi] =$ $ \top \Rightarrow$ $ [M\models\phi]_t = \top$;
\item $[M\models\phi] = \bot \Rightarrow [M\models\phi]_t = \bot$.
\end{enumerate*}
That is, a formula which is true (false) under the three-valued semantics is also true (false) under the thorough semantics.

There exists a subset of LTL formulae, known in the literature as \emph{self-minimizing}~\cite{godefroid2005MCvsGMC}, such that the two semantics coincide. Formally,
given a model $M$ and a  self-minimizing LTL property $\phi$, then $[M\models\phi]=[M\models\phi]_t$. It has been observed that most practically useful LTL formulae belong to this subset~\cite{godefroid2005MCvsGMC}.

%In~\cite{godefroid2005MCvsGMC} the authors showed that many temporal-logic formulae of practical interest are self-minimizing.
%More precisely, all the formulae that do not contain atomic propositions with mixed polarity (negated and not negated) in their negation normal form are self-%minimizing; if the formulae follow specific patterns are self-minimizing;
%some LTL formulae that are not self-minimizing can be converted into self-minimizing LTL formulae using a procedure called semantic-minimization.
%The interested reader may refer to~\cite{godefroid2005MCvsGMC} for more details.
%
%


We present a \emph{model checking} algorithm for PKSs and LTL formulae based on three-valued semantics. 
%This algorithm exploits two classical model checking procedures for KSs.
This procedure considers a version of $M$, called \emph{complement-closed}~\cite{bruns2000model}, in which for every proposition $p \in AP$, there exists a new proposition $\overline{p}$, called complement-closed proposition, such that $L(s,\overline{p})=$ comp$(L(s,p))$, for all $s \in S$.
%	The proposition $q$, whose value is complementary to the one of $p$, is indicated as $\overline{p}$.
For example, the complement-closed version of the PKS of the semaphore example is presented in Figure~\ref{fig:model}.
%, where the atomic propositions $\overline{g}$ and $\overline{r}$ are associated with the complement of the values of  propositions green ($g$) and red ($r$), respectively.

\begin{figure}[t]
\begin{minipage}[b]{.5\textwidth}
  \begin{figure}[H]
 \centering
\begin{tikzpicture}[scale=0.33] %[x={10.0pt},y={10.0pt}]

\node [state, initial, initial text={},initial where=above](s_0)
[label={
[align=left,yshift=0.1cm]
above:
$\boldsymbol{\overline{g}=\top}$\\
$\boldsymbol{\overline{r}=\bot}$\\
$g=\color{black}\bot$\\
$r=\top$}] 
{$s_0$};   

\node[state] (s_1) 
[left=of s_0,
label={
[align=left,yshift=0.1cm]
above:
$\boldsymbol{\overline{g}=\bot}$\\
$\boldsymbol{\overline{r}=\top}$\\
$g=\color{black}\top$\\
$r=\bot$}]  
{$s_1$};

\node[state] (s_2) 
[right=of s_0,
label={
[align=left,yshift=0.1cm]
above:
$\boldsymbol{\overline{g}=?}$\\
$\boldsymbol{\overline{r}=?}$\\
$g=\color{black}?$\\
$r=?$}]  
{$s_2$};


\path[->]      
(s_0) edge [bend left] node [above]{} (s_1) 
(s_1) edge [bend left] node [above]{} (s_0)
(s_0) edge [bend left] node [above]{} (s_2)
(s_2) edge [bend left] node [above]{} (s_0)   

\end{tikzpicture}



\caption{The PKS of  the  crossing semaphore.}
\label{fig:model}
\end{figure}
\end{minipage}
\begin{minipage}[b]{.5\textwidth}
  \begin{figure}[H]
\centering
\begin{tikzpicture}[scale=0.33] %[x={10.0pt},y={10.0pt}]

\node[state, initial, initial text={}] (q_0) 
[label={
[align=left,yshift=-0.1cm]
below:
$\eta(q_0)= \LTLnext \LTLfinally \LTLglobally \lnot g$\\
$\mu(q_0)= \LTLnext \LTLglobally \LTLfinally g$}]
{$q_0$};
   
\node[state,accepting] (q_1) 
[right = 2.1cm of q_0,
label={
[align=left,yshift=-0.1cm]
below:
$\eta(q_1)= \lnot g \LTLand \LTLnext \LTLglobally \lnot g  $\\
$\mu(q_1)= g \LTLor \LTLnext \LTLfinally g$}]
{$q_1$};

\path[->]      
(q_0) edge [loop above]node [above]{$true$} (q_0) 
(q_0) edge node [above,align=center]
{\\ \\ \\ $\overline{g}$} (q_1)
(q_1) edge [loop above]node [above right, pos=0.8,align=center]
{\\ \\ \\ $\overline{g}$} (q_1);    

\end{tikzpicture} 
\caption{The BA associated with $\phi_2$.}
\label{fig:property}
\end{figure}
\end{minipage}
\end{figure}



%
%\begin{definition}[PKS optimistic and pessimistic labelling~\cite{bruns2000model}]
%Given a three-valued labelling function $L$,  we define the derived optimistic ($L_{opt}$) and pessimistic ($L_{pes}$) labelling function for every state s of the model $M$ as follows: 
%\begin{equation}
%    L_{opt}(s,p) \buildrel \text{def}\over = 
%                					\begin{cases}
%                  						\top & \quad  \text{if}\ L(s,p)=?\\
%                  						L(s,p) &  \quad otherwise
%                					\end{cases}
%\end{equation}            					  
%\begin{equation}      					  
%   L_{pes}(s,p) \buildrel \text{def}\over = 
%	                            \begin{cases}
%                  				       \bot  & \quad \text{if}\ L(s,p)=?\\
%                  						L(s,p) & \quad otherwise
%                					\end{cases}
%\end{equation}
%\end{definition}


The model checking procedure for a PKS $M$ is based on an optimistic and pessimistic approximation of $M$'s complement-closure.
The optimistic (pessimistic) approximation function $L_{opt}$  ($L_{pes}$) associates the value $\top$ ($\bot$) to each atomic proposition of the complement-closure of $M$ with value $?$.
Given a PKS $M=\langle S,R, S_0, L \rangle$, we have $M_{pes}=\langle S,R, S_0, L_{pes} \rangle$ for the pessimistic case, and $M_{opt}=\langle S,R, S_0, L_{opt} \rangle$ for the optimistic one.
%
%The model checking procedure exploits the under and the over approximations as we show hereafter.

The three-valued model checking algorithm assumes that property $\phi$ is rewritten using complement-closed propositions. 
The procedure works in two steps.
First, the formula is expressed such that negations only appear in front of atomic propositions. 
Second, each negated proposition is substituted by the corresponding complemented proposition.
Let $\phi$ be an LTL formula obtained using the procedure just discussed, $M=\langle S, R, S_0, L\rangle $ a PKS with $s \in S$, and $M_{pes}$ and $M_{opt}$ the corresponding pessimistic and optimistic cases. 
%
%Then, $[(M,s)\models\phi]$ has been defined in~\cite{bruns2000model}
\setcounter{footnote}{0}
Then, \cite{bruns2000model}\footnote{In~\cite{bruns2000model} the procedure is presented for PML but is valid also for LTL (see \cite{bruns2000model,godefroid2005MCvsGMC,godefroid2011ltl}).} has defined:
\begin{equation}
    [(M,s)\models\phi]  = 
                						 \begin{cases}
                  				 				\top & \quad \text{if}\ (M_{pes},s)\models\phi\\
                  								\bot & \quad \text{if}\ (M_{opt},s)\not\models\phi\\
                  								? & \quad  otherwise
                						\end{cases} \nonumber
\end{equation}

This technique exploits two runs of the classical two-valued model checking performed  on a pessimistic and an optimistic completion of  $M$.
%Note that whenever a $\top$ or $\bot$ value is returned, the property $\phi$ is guaranteed to be true/false in all the completions of $M$.
%For additional details on this procedure see~\cite{bruns2000model,godefroid2011ltl}.




%\begin{theorem}(The verification procedure is correct)
%\end{theorem}
%\begin{proof}
%Given a self-minimizing temporal logic formula $\phi$, by Proposition~\ref{def:self-minimizing}, generalized (which is based on the thorough semantic) and
%three-valued model checking produce the same result.
%Thus, the considered three-valued model checking algorithm returns the correct value also under the thorough interpretation.
%\end{proof}




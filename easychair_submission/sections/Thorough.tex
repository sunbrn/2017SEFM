\subsection{Thorough semantics and \NAME}
As stated in Section~\ref{sec:preliminaries},  three-valued semantics does not always behave in accordance with the natural intuition~\cite{bruns2000model}.
When $\phi$ possibly holds in $M$,   it is desirable that there exist two completions $M^\prime$ and $M^{\prime\prime}$ of $M$  such that $M^\prime$ satisfies  $\phi$  and   $M^{\prime\prime}$ violates $\phi$.
This property is not ensured by the three-valued semantics, and is the motivation that leads to introduce thorough LTL semantics. Hereafter, we discuss how the adoption of thorough semantics would affect the use of the \NAME\ framework.

Given a PKS  $M$ and a property  $\phi$,  \NAME\ produces the following outputs:

\noindent \emph{Property is satisfied.}
%In this case, \NAME\ works correctly. 
\NAME\ works correctly. 
A property $\phi$ that evaluates to $\LTLtrue$ under three-valued semantics is also satisfied under thorough semantics.
Thus, the  verification result is correct.
Also the proof is correct since it shows that any completion of $M$ satisfies $\phi$. 

\noindent \emph{Property is not satisfied.}
%In this case, \NAME\ works correctly. 
 \NAME\ works correctly. 
When the model checker returns a $\LTLfalse$ value, the counterexample shows a behavior that violates $\phi$.
A property $\phi$  that is not satisfied considering the three-valued semantics, is also not satisfied considering the thorough semantics.
%Thus, the counterexample is a correct counterexample that proves the existence of a completion of $M$ that violates $\phi$.
Thus, the counterexample is correct and proves the existence of a completion of $M$ that violates $\phi$.
 
\noindent \emph{Property is possibly satisfied.}
%This case is not handled by \NAME\ correctly for all LTL properties.
 \NAME\ does not work correctly  for all LTL properties. 
When the three-valued model checker returns  $?$  the property is possibly satisfied considering the three-valued semantics but no conclusion can be drawn based on thorough semantics.
Indeed, there are cases in which a $?$ is returned, but all the completions of the model either satisfy or do not satisfy $\phi$. 
The computed counterexample and proof can be spurious under the thorough semantics.


%In this case, if a correct result is required, the use of a generalized model checking procedure~\cite{bruns2000model} (not discussed here) becomes necessary. 
%Since when a property $\phi$ is evaluated to $?$ there could exist a completion that violates $\phi$, the produced proof is not correct.

%
%
%Correctness is analyzed by comparing the results returned by \NAME\ with the  results  that would be returned considering the thorough LTL semantics.
%We stated that a model checking result---the verification result, the counterexample or the possible counterexample---is \emph{correct} (\validCounterexample ) if it corresponds to the results that would be obtained considering the thorough LTL semantic.
%Otherwise, it is \emph{incorrect} (\spuriosCounterexample), i.e., the counterexample can be spurious.
%We stated that a proof generated by \NAME\ is \emph{correct} (\validProof) if it proves the satisfaction/possibly satisfaction of a property considering the thorough LTL semantics. 
%Otherwise, it is \emph{incorrect} (\notvalidProof ).
%Table~\ref{tab:results} shows the comparison. 
%The rows of the table specify whether \NAME\ states that the property of interest is satisfied ($\LTLtrue$), possibly satisfied ($?$) or not satisfied ($\LTLfalse$).
%If the property is satisfied/possibly satisfied Table~\ref{tab:results} considers the correctness of the verification result ($\LTLtrue$, $?$ or $\LTLfalse$) and the proof.
%If the property is not satisfied only the correctness of the verification result is considered since no proof is produced.
%The correctness of the framework w.r.t. LTL properties is presented by the column with labeled as LTL.

%\textbf{Property satisfied.}
%When a property $\phi$ is evaluated to $\LTLtrue$ considering the three-valued semantics, it is also satisfied considering the thorough semantics.
%Thus, the  verification result is correct.
%Also the proof is correct since it shows that any completion of $M$ satisfies $\phi$. 
%
%\textbf{Property not satisfied.}
%When the model checker returns a $\LTLfalse$ value, the counterexample shows a behavior that violates $\phi$.
%A property $\phi$  that is not satisfied considering the three-valued semantics, is also not satisfied considering the thorough semantics.
%Thus, the counterexample is a correct counterexample that proves the existence of a completion of $M$ that violates $\phi$.
% 
%\textbf{Property possibly satisfied.}
%When the three-valued model checker returns a $?$ value the property is possibly satisfied considering the three-valued semantics but no conclusion can be achieved by considering the thorough semantics.
%Indeed, there are cases in which a $?$ is returned but all the completions of the model either satisfy or do not satisfy the property of interest. 
%The computed counterexample can be a spurious counterexample under the thorough semantics.
%The corresponding cell in Table~\ref{tab:results} is marked with a \spuriosCounterexample\ symbol meaning that the counterexample is not correct.
%In this case, if a correct result is required, the use of a generalized model checking procedure~\cite{bruns2000model} (not discussed here) becomes necessary. 
%Since when a property $\phi$ is evaluated to $?$ there could exist a completion that violates $\phi$, the produced proof is not correct.


%
%\begin{table}[t]
%\centering
%\caption{LTL and self-minimizing LTL validity of the framework outputs.}
%\label{tab:results}
%\begin{tabular}{   c  c | c | c | }
%\cline{1-4}
%  \multicolumn{1}{| c }{Result} & \multicolumn{1}{| c |}{Type of Output} & \multicolumn{1}{c |}{LTL} & \multicolumn{1}{c |}{Self-minimizing LTL} \\
%\cline{1-4} 
%  \multicolumn{1}{| c }{\multirow{2}{*}{$\LTLtrue$}} &  \multicolumn{1}{| c |}{Verification result} & \multicolumn{1}{ c |}{\validCounterexample}  & \validCounterexample  \\
%    \cline{2-4}
% \multicolumn{1}{| c }{} &  \multicolumn{1}{| c |}{Proof} & \validProof  & \validProof  \\
%  \cline{1-4}
%  \multicolumn{1}{| c }{\multirow{2}{*}{$?$}} &  \multicolumn{1}{| c |}{Verification result}  & \multicolumn{1}{ c |}{\spuriosCounterexample} & \validCounterexample  \\
%  \cline{2-4}
%  \multicolumn{1}{| c }{} &  \multicolumn{1}{| c |}{Proof} & \notvalidProof\  & \validProof  \\
%   \cline{1-4}
% \multicolumn{1}{| c }{$\LTLfalse$} & \multicolumn{1}{| c |}{Verification result} &  \multicolumn{1}{ c |}{\validCounterexample} & \validCounterexample   \\
% \hline
%\end{tabular}
%\end{table}
% 
%
\textbf{Example.}
\emph{The results obtained for $\phi_1$ and $\phi_3$ of the crossing semaphore example are correct both considering the three-valued and the thorough semantics. 
Since $\phi_1$ is satisfied, the proof is a correct proof that justifies why all the completions of the model presented in Figure~\ref{fig:modelmot} satisfy $\phi_1$.
The counterexample  returned for $\phi_3$ is correct, i.e., all the completions of the model presented in Figure~\ref{fig:modelmot} exhibit the behavior returned as a counterexample.}


\vskip 0.05in  
\textbf{Self-minimizing LTL formulae.} 
Self-minimizing LTL formulae are a subset of LTL formulae that present an interesting property:  three-valued and  thorough semantics are equivalent, i.e., if $\phi$ is self-minimizing, then $[(M, s) \models \phi]=[(M, s) \models \phi]_t$.
Therefore, the three-valued model checking framework presented in Section~\ref{sec:preliminaries} produces a result that is correct also considering the thorough semantics. 
For this reason, whenever the three-valued model checker returns $?$,  the proof and the possible counterexample produced by \NAME\ are also correct  under the thorough semantics. In~\cite{godefroid2005MCvsGMC}, the authors propose a first grammar for this LTL subset. 
The grammar does not capture entirely this set. However, it can be used to generate formulae that are self-minimizing by construction, or to check whether a formula is self-minimizing (sufficient condition). 
Furthermore, the authors argue that the set of self-minimizing LTL formulae contains most property patterns of practical interest, such as absence, universality, existence, response and response chain~\cite{dwyer1998property}. 
%For these reasons it is  possible in practice to use the version of \NAME\ of Figure~\ref{Fig.3vdvinstance} also under the thorough semantics interpretation.
For these reasons it is  possible in practice to use the version of \NAME\ of Figure~\ref{Fig.3vdvinstance} also considering the thorough semantics.


\textbf{Example}
\emph{Property $\phi_2$ is a special instance of LTL response pattern which, according to~\cite{godefroid2005MCvsGMC}, is self-minimizing.
Thus, the possible counterexample and the proof returned by \NAME\ are correct.}
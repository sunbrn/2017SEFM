This work presented \NAME, a theoretical framework for a correct integration of existing multi-valued model checkers and theorem provers. 
Whenever the property of interest is definitely satisfied, or possibly satisfied, \NAME\ provides  information regarding why a certain result is returned by the model checker.
The proof gives intuition on what is working correctly in the current design and insights for the next development rounds.
We instantiate \NAME\ considering a PKS, to express the model of the system, and LTL, to specify the property of interest. 
We show that the instantiation is feasible and sound, and requires changing the model checking algorithm to accomodate the execution of the theorem prover.
\NAME\ has been evaluated considering a safety critical example~\cite{arcaini2015formal}, which showed  the effectiveness  of the approach. 
We also discussed the applicability of the approach in real world cases.

As future work, we aim to implement \NAME\ by integrating existing model checkers and theorem provers. 
This  will allow us to provide further evidence of the impact of \NAME\ in continuous system development and to analyze the challenges of realistic systems.
We would like to introduce possible extensions of the currently considered formalisms: other forms of partial systems models and other multi-valued logic options for the properties.
Finally, we also wish to investigate thoroughly how the proofs can be written in the most understandable and useful form for the designer.


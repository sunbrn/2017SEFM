\subsection{Adapting the theorem prover.}
\label{sec:adapting}
The deductive verification framework presented in~\cite{peled2001model}  exploits the product between a state labeled transition system and  a GBA \gba\ obtained by $\lnot\phi$ to generate the proof.
To enable the algorithm to work on KSs and BAs, 
we describe
%it is necessary to specify
 how to associate LTL formulae with each state of the BA and how to identify failed states of the product automaton.

\vskip 0.05in  
\textbf{Identification of the formulae that hold in the states of  the BA.}
We assume that the degeneralization procedure~\cite{clarke1999model}, that converts the GBA  \gba\
%$\mathcal{G}$ 
 into an equivalent BA \ba\
 %$\mathcal{A}$,
  behaves as follows: when a new state $q$ of 
  %$\mathcal{A}$
  \ba\ is created from a state $q^\prime$ of 
  \gba ,
  %$\mathcal{G}$, 
the formulae $\eta(q^\prime)$ and $\mu(q^\prime)$ are also associated to $q$.

\vskip 0.05in  
\textbf{Identification of failed states.} 
Following the procedure mentioned in Section~\ref{sec:preliminaries}, the product automaton 
%$M\otimes\mathcal{A}$
$M\otimes \ba$
 between the KS $M$ and the BA 
 %$\mathcal{A}$ 
\ba\ is modified to also generate \emph{failed states}.
Specifically, the product is computed using the rules~\ref{eq:classicalTransition} and~\ref{eq:failedTransition}.
%
\vspace{-1cm}
\begin{multicols}{2}
\begin{equation}
\label{eq:classicalTransition}
 \inferrule{s\rightarrow t \land q \xrightarrow{L(t)}p}{\langle s,q \rangle \rightarrow \langle t,p \rangle}  
\end{equation}\break
\begin{equation}
\label{eq:failedTransition}
 \inferrule{s\rightarrow t \land q \xrightarrow{\cancel{L(t)}}p}{\langle s,q \rangle \dashrightarrow \langle t,p \rangle}  
\end{equation}
\end{multicols}
%
Rule~\ref{eq:classicalTransition} is the classical rule used to compute the product automaton. 
It specifies that the state of the product $\langle s,q \rangle$  moves to $\langle t,p \rangle$ only if the transition $q \xrightarrow{L(t)}p$ that moves the BA from  $q$ to $p$ has the same label of the state $t$ of $M$.
Rule~\ref{eq:failedTransition} specifies how to compute failed states.
It states that the failed state $\langle t,p \rangle$ is generated in the product  when a transition  that moves the BA 
%$\mathcal{A}$
 \ba\ from $q$ to $p$ is labelled differently with respect to the state $t$ reached by the model $\mathcal{M}$ when the transition $s\rightarrow t$ is fired.
This is indicated using the notation $q \xrightarrow{\cancel{L(t)}}p$.
For this reason, the transition $\langle s,q \rangle \dashrightarrow \langle t,p \rangle$  from $\langle s,q \rangle $ to $\langle t,p \rangle$ is dashed.
Let us consider the product presented in Figure~\ref{fig:productOpt}  computed from the KS $M_{opt}$ obtained from the PKS in Figure~\ref{fig:model} and the BA of Figure~\ref{fig:property}.
The transition $\langle s_0, q_0 \rangle$ to  $\langle s_1, q_1 \rangle$ of the product presented in Figure~\ref{fig:productOpt} is dashed, since the proposition $\overline{g}$ is false in $s_1$, while the labeling of the transition from $q_0$ to $q_1$ requires $\overline{g}$ to be true for the transition to be performed.

The set 
%$\mathcal{F}(M\otimes\mathcal{A})$ 
 $\mathcal{F}(M\otimes \ba)$ of the \emph{failed states} contains the states $\langle t, p \rangle$ obtained by applying rule~\ref{eq:failedTransition}.
Note that, as stated in Section~\ref{sec:preliminaries}, each failed state $\langle s, q \rangle$ is such that $s \models \mu(q) $.
For example, the state $\langle s_1, q_1 \rangle$ of the product  presented in Figure~\ref{fig:productOpt} is a failed state.
Indeed, $s_1$ satisfies the property $\mu(q_1)= g \LTLor \LTLnext \LTLfinally g$ associated with the state $q_1$.



%%%%%%%%%%%%%%
\begin{theorem}
\label{th:deductivecorrecteness}
The deductive verification procedure is correct.
\end{theorem}

\begin{proof}
We show that the states identified as \emph{failed}  correspond to the ones that would be identified using~\cite{peled2001model}. 
In~\cite{peled2001model}, a state $\langle t, p \rangle$ is failed if the propositional assignment of $t$ does not satisfy the conditions specified in the state $p$. 
It is well known~\cite{gerth1996ltl2ba,clarke1999model}, that a GBA \gba\ associated with $\phi$ is such that
\begin{enumerate*}[label={(\arabic*)}]
\item all the transitions $(q,\alpha, p) \in \Delta$ that reach a state $p$ of the GBA have the same label $\alpha$ and that
\item a transition $(q,\alpha, p) \in \Delta$ is in the GBA  if and only if $\alpha$ satisfies the conjunction of the negated and non negated propositions that hold in the state $p$.
\end{enumerate*}
By construction, the latter of these properties also holds in the BA obtained from the GBA by applying the degeneralization procedure~\cite{clarke1999model}.
Thus, since all the transitions that reach $p$ are labelled with $\alpha$, a transition $\langle s,q \rangle \dashrightarrow' \langle t,p \rangle$ is added to the product automaton if and only if the propositional assignment of $t$ does not satisfy the propositional assignment specified in the state $p$. 
Furthermore, BAs acceptance condition is a special case of fairness condition used in~\cite{peled2001model}. 
Thus, the proposed deductive verification procedure is a special case of~\cite{peled2001model}, with regard to acceptance.\qed
\end{proof}

\begin{figure}[t]
\begin{minipage}[b]{.5\textwidth}
  \begin{figure}[H]
\centering




\begin{tikzpicture}[scale=0.33] %[x={10.0pt},y={10.0pt}]

\node[statep, initial, initial text={}, initial where=above] (s0q0) 
%[label={[yshift=-1.1cm,xshift=0cm]\scriptsize $\{q_0\}$},font=\scriptsize] {$\langle s_0,q_0\rangle$}; 
[font=\scriptsize] {$\langle s_0,q_0\rangle$}; 
\node[statep,  initial, initial text={}, initial where=above] (s0q1) 
%[below = 0.6cm of s0q0,label={[yshift=-1.1cm,xshift=0.5cm]\scriptsize $\{q_1\}$},font=\scriptsize] {$\langle s_0,q_1\rangle$}; 
[below = 0.6cm of s0q0,font=\scriptsize] {$\langle s_0,q_1\rangle$}; 

\node[statep] (s1q0) 
%[left = 0.4cm of s0q0,label={[yshift=-1.1cm,xshift=0cm]\scriptsize $\{q_0\}$},font=\scriptsize] {$\langle s_1,q_0\rangle$}; 
[left = 0.4cm of s0q0,font=\scriptsize] {$\langle s_1,q_0\rangle$}; 
\node[statep] (s1q1) 
%[below = 0.6cm of s1q0,label={[yshift=-1.1cm,xshift=0.5cm]\scriptsize $\{q_1\}$},font=\scriptsize] {$\langle s_1,q_1\rangle$}; 
[below = 0.6cm of s1q0,font=\scriptsize] {$\langle s_1,q_1\rangle$}; 

\node[statep] (s2q0) 
%[right = 0.4cm of s0q0,label={[yshift=-1.1cm,xshift=0cm]\scriptsize $\{q_0\}$},font=\scriptsize] {$\langle s_2,q_0\rangle$}; 
[right = 0.4cm of s0q0,font=\scriptsize] {$\langle s_2,q_0\rangle$}; 
\node[statep] (s2q1) 
%[below = 0.6cm of s2q0,label={[yshift=-1.1cm,xshift=-0.5cm]\scriptsize $\{q_1\}$},font=\scriptsize] {$\langle s_2,q_1\rangle$}; 
[below = 0.6cm of s2q0,font=\scriptsize] {$\langle s_2,q_1\rangle$}; 

%%LABELS for steps
\node[empty] (step3u) [above = 0.155cm of s0q0] {}; 
\node[empty] (step1u) [below = 0.155cm of s0q1] {}; 
\node[emptynode] (step3) [left = 1.9cm of step3u] {\tiny Step 3}; 
\node[emptynode] (step1) [left = 1.9cm of step1u] {\tiny Step 1}; 
\node[emptynode] (step1b) [right = 1.9cm of step1u] {\tiny Step 1}; 
\node[emptynode] (step2) [right= 1.1cm of step1] {\tiny Step 2};


\path[->]      
(s0q0) edge [bend left=15] node []{} (s1q0) 
(s1q0) edge [bend left=15] node []{} (s0q0)  

(s0q0) edge [bend left=15] node []{} (s2q0) 
(s2q0) edge [bend left=15] node []{} (s0q0)  

(s1q0) edge [bend right=5] node []{} (s0q1)
(s2q0) edge [bend left=5] node []{} (s0q1)

(s0q0) edge [bend left=5,dashed] node []{} (s1q1)
(s0q0) edge [bend right=5,dashed] node []{} (s2q1)

(s0q1) edge [bend left=5,dashed] node []{} (s1q1) 
(s0q1) edge [bend right=5,dashed] node []{} (s2q1) 
;    

\begin{pgfonlayer}{background}

%FAILED
\filldraw [line width=2mm,join=round,black!5]
(s1q1.south -| s1q1.west) rectangle (s1q1.north -| s1q1.east);
\filldraw [line width=2mm,join=round,black!5]
(s2q1.south -| s2q1.west) rectangle (s2q1.north -| s2q1.east);

%SUCCESSORS
\filldraw [line width=2mm,join=round,black!20]
(s0q1.south -| s0q1.west) rectangle (s0q1.north -| s0q1.east);

%INDUCTION
\filldraw [line width=2mm,join=round,black!35]
(s1q0.south -| s1q0.west) rectangle (s2q0.north -| s2q0.east);

\end{pgfonlayer}


\end{tikzpicture}

\caption{Product $I_{opt}=M_{opt}\otimes\mathcal{A}_{\lnot\phi_2}$}
\label{fig:productOpt}
 \end{figure}
\end{minipage}%
\begin{minipage}[b]{.5\textwidth}
  \begin{figure}[H]
\centering
\begin{tikzpicture}[scale=0.33] %[x={10.0pt},y={10.0pt}]

\node[statep, initial, initial text={}, initial where=above] (s0q0) 
%[label={[yshift=-1.1cm,xshift=0cm]\scriptsize $\{q_0\}$},font=\scriptsize] {$\langle s_0,q_0\rangle$}; 
[font=\scriptsize] {$\langle s_0,q_0\rangle$}; 
\node[statep,  initial, initial text={}, initial where=above] (s0q1) 
%[below = 0.6cm of s0q0,label={[yshift=-1.1cm,xshift=0cm]\scriptsize $\{q_1\}$},font=\scriptsize] {$\langle s_0,q_1\rangle$}; 
[below = 0.6cm of s0q0,font=\scriptsize] {$\langle s_0,q_1\rangle$}; 

\node[statep] (s1q0) 
%[left = 0.4cm of s0q0,label={[yshift=-1.1cm,xshift=0cm]\scriptsize $\{q_0\}$},font=\scriptsize] {$\langle s_1,q_0\rangle$}; 
[left = 0.4cm of s0q0,font=\scriptsize] {$\langle s_1,q_0\rangle$}; 

\node[statep] (s2q0) 
%[right = 0.4cm of s0q0,label={[yshift=-1.1cm,xshift=0cm]\scriptsize $\{q_0\}$},font=\scriptsize] {$\langle s_2,q_0\rangle$}; 
[right = 0.4cm of s0q0,font=\scriptsize] {$\langle s_2,q_0\rangle$}; 
\node[statep] (s2q1) 
%[below = 0.6cm of s2q0,label={[yshift=-1.1cm,xshift=0cm]\scriptsize $\{q_1\}$},font=\scriptsize] {$\langle s_2,q_1\rangle$}; 
[below = 0.6cm of s2q0,font=\scriptsize] {$\langle s_2,q_1\rangle$}; 

\path[->]      
(s0q0) edge [bend left=15] node []{} (s1q0) 
(s1q0) edge [bend left=15] node []{} (s0q0)  

(s0q0) edge [bend left=15] node []{} (s2q0) 
(s2q0) edge [bend left=15] node []{} (s0q0)  

(s1q0) edge [bend right=5] node []{} (s0q1)
(s2q0) edge [bend left=5] node []{} (s0q1)

%(s0q0) edge [bend left=5,dashed] node []{} (s1q1)
(s0q0) edge [bend right=5] node []{} (s2q1)

%(s0q1) edge [bend left=5,dashed] node []{} (s1q1) 
(s0q1) edge [bend left=15] node []{} (s2q1) 
(s2q1) edge [bend left=15] node []{} (s0q1) 
;    

\end{tikzpicture} % pic 1
\caption{Product $I_{pes}=M_{pes}\otimes\mathcal{A}_{\lnot\phi_2}$}
\label{fig:productPess}
 \end{figure}
\end{minipage}%
\end{figure}